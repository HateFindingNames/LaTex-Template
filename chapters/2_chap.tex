% Im Theorieteil ist der aktuelle Stand der Technik/Forschung wertungsfrei darzustellen. Er dient dazu, den wissenschaftlichen Kontext für die Aufgabenstellung herauszuarbeiten. Dementsprechend sollte die Darstellungsweise nicht zu allgemein gewählt werden, sondern nur die für die Aufgabenstellung relevanten Theorien und notwendigen Definitionen dargelegt werden. Stehen mehrere Theorien zur Verfügung, ist ein Überblick zu geben. Definitionen sind nur dann notwendig, wenn sie nicht ohnehin als allgemein, d.h. von Vertreter des Fachbereichs, bekannt vorausgesetzt werden können und für den weiteren Verlauf der Arbeit relevant sind. Die Theorie hat einen entscheidenden Einfluss auf die Methodenwahl und die Interpretation der Ergebnisse. Der Stand der Technik ist u.a. dokumentiert in Monografien, Vortragsmanuskripten, Patenten und in besonderer Weise in Artikeln wissenschaftlicher Zeitschriften. Diese Quellen Seite 5 sind zu nutzen. Es wird eine Analyse von mindestens vier Zeitschriftenartikeln erwartet, von denen mindestens einer in englischer Sprache verfasst ist. Bei der Auswahl der Literatur ist auf Aktualität der Arbeiten zu achten.
\chapter{Theorie}
\nomenclature[A]{SCAB}{Some Cops Are Brutal}%
\nomenclature[G]{\(\alpha\)}{Manchmal ein Winkel, manchmal auch Winkelbeschleunigung}%
\nomenclature[L]{F}{Oft das Zeichen für Kraft\nomunit{\newton}}%
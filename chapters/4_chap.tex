% Ausgehend von den in der Literaturbesprechung identifizierten Defiziten im Stand der Technik wird in diesem Teil erklärt, wie und warum ein bestimmtes Problem ausgewählt wurde, wer es als problematisch empfindet und welche Bedeutung die Problematik hat. Es sollte dargestellt werden, welchen Beitrag die Bearbeitung des Themas einerseits zur Forschung (Theorie, Modell, Methoden, Fakten) leistet und andererseits welche praktische Relevanz mit dem Thema verbunden ist. Da oft nicht alle Aspekte eines Problems im Rahmen einer wissenschaftlichen Arbeit behandelt werden können, wird die Problemstellung durch eine konkretisierte Zielsetzung eingegrenzt. Sie ergibt sich aus dem Forschungsinteresse sowie Erwägungen zur Machbarkeit und zum vertretbaren Aufwand. Kurz und klar sollte aufgeführt werden, was erreicht werden soll, welche Ergebnisse zu welchem Verwendungszweck angestrebt werden, und welche Art von Schlussfolgerungen in Bezug auf das Gesamtproblem daraus möglich werden sollen.
\chapter{Aufgabenstellung}
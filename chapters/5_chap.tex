% In diesem Kapitel soll der Untersuchungsplan zur Bearbeitung der Problemstellung beschrieben werden. Dies beinhaltet Verfahrensweisen und Bearbeitungsschritte, die zur Zielerreichung führen: Was soll auf welchem Weg, wo, wann, in welchen Situationen durch wen ermittelt werden? Es sollte außerdem nachvollziehbar sein, warum bestimmte Methoden verwendet werden und andere nicht (z.B. Abwägung der Vor-/Nachteile in Bezug auf das Thema). Die Komplexität des Untersuchungsplans beeinflusst den Umfang der Arbeit. Bei der Durchführung einer experimentellen Studie, sollte die Grundgesamtheit der untersuchten Fälle oder Personen in allen relevanten Merkmalen so detailliert wie nötig beschrieben werden. Dies gilt in besonderem Maße für die verwendete Stichprobe (bzw. Teilmenge der Grundgesamtheit), weil sie über die Aussagefähigkeit der Untersuchung entscheidet. Ebenso sollte begründet werden, warum die gezogene Stichprobe angemessen
% ist. Wissenschaftliche Kriterien:
% • Die verwendeten Methoden sind unter Vorgabe einer möglichst hohen Qualität hinsichtlich Objektivität, Reliabilität und Validität auszuwählen.
% • Die Durchführung der Untersuchung umfasst die methodisch-organisatorischen Details der Datenerhebung. Diese Beschreibung muss anderen Forschern ermöglichen, die Untersuchung zu wiederholen.
% • Die Auswertungsverfahren sind nur dann ausführlicher darzustellen, falls sie nicht allgemein üblich und bekannt sind (z. B. Eigenentwicklung eines statistischen Verfahren).
% • Forschungsethische Implikationen der Untersuchung müssen bedacht und eventuelle Aushandlungen durchgeführt werden: Wem nützt/schadet die Untersuchung? Welche Rechte haben untersuchte Personen/MitarbeiterInnen?
\chapter{Methodik}
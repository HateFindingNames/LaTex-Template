\usepackage{microtype}
\usepackage[reqno,intlimits]{amsmath} % [reqno] um die gleichungsnummerierung rechts zu haben
% intlimits: Grenzen für Integrale unterhalb und oberhalb des Zeichens
\usepackage{amssymb}
\usepackage{array}
\ifbool{deutsch}{\usepackage[ngerman]{babel}}{\usepackage[english]{babel}}
\usepackage{varioref}
\usepackage[T1]{fontenc}
\usepackage[utf8]{inputenc}
\usepackage{booktabs}
\usepackage{calc}
\usepackage{cancel} % Place lines through maths formulae
\usepackage[labelfont={footnotesize,sf,bf},textfont={footnotesize,sf}]{caption} %Format (Textgröße, Textform) für Bildtext 
%normalsize
%scriptsize
% sc --> smallcaps
% bf --> bold face
% sf --> sans serif
\usepackage[table]{xcolor}
\usepackage[right]{eurosym}
\usepackage{ellipsis}
\usepackage{graphicx}
\usepackage{float}
\usepackage{gensymb}
\usepackage{csquotes}
\usepackage{listings}
\usepackage{longtable}
\usepackage{lastpage}
\usepackage{lscape}
\usepackage{lmodern} % Silbentrennung
\usepackage{makeidx}
\usepackage{multirow}
\usepackage{multicol}
\usepackage[intoc]{nomencl} % zwei Spalten beim Formelzeichenverzeichnis
\usepackage{nicefrac}
\usepackage{paralist} % This package provides some new list environments. Itemized and enumerated lists can be typeset within paragraphs, as paragraphs and in a compact version. Most environments have optional arguments to format the labels. Additionally, the LATEX environments itemize and enumerate can be extended to use a similar optional argument.
\usepackage{pdfpages} % https://www.ctan.org/pkg/pdfpages
% Define user colors using the RGB model
%\usepackage{colortbl}
%\definecolor{dunkelgrau}{rgb}{0.8,0.8,0.8}
%\definecolor{hellgrau}{rgb}{0.95,0.95,0.95}
\usepackage[figuresright]{rotating}
\usepackage{scrlayer-scrpage}
\usepackage[locale=DE,per-mode=fraction]{siunitx}
% \usepackage[font={scriptsize,sl},captionskip=3pt]{subfig}
\usepackage{subcaption}
\usepackage{shortvrb}
\usepackage{tablefootnote}
\usepackage{tabularx}
\usepackage{tabulary}
\usepackage{textcomp}
\usepackage{tocbasic}
% \usepackage{tikz}
\usepackage{times}
\usepackage{xurl}
\usepackage{wrapfig}
\usepackage{hyperref} % muss am Schluss stehen
\hypersetup{
    colorlinks=true, % Color links instead of disgusting boxes :barfing_emoji:
    linkcolor={black}, % Color of internal links
    citecolor={black}, % Color of citations
    urlcolor={blue} % Color of external hyperlinks
}
%\usepackage{unicode-math}
\usepackage[nonumberlist, acronym, toc, section]{glossaries} % muss nach hypersetup stehen
%\usepackage{romannum} % Seitenzahlen in römischen Ziffern
\usepackage{scrhack}
\usepackage[style=numeric, citestyle=numeric, backend=biber]{biblatex}
\usepackage[useregional]{datetime2}
\usepackage{cleveref}
% \usepackage{isotope} % um chemische gleichungen hübscher darstellen zu können
\usepackage{framed} % rahmen um dinge malen können
\usepackage{svg} % um auch vektorgrafiken als bild einfügen zu können
\usepackage{adjustbox} % skaliert floats dynamischer (anti-over/underfull)
\usepackage[numbered]{bookmark} % platziert im pdf reader bei der kapitelübersicht die jeweiligen kapitelnummer vor die kapitelüberschriften
%
% ---------- Make floats stay within the chapter/section/subsection they got placed ------------------
\usepackage{placeins}
\let\Oldsection\section
\renewcommand{\section}[1]{\FloatBarrier\Oldsection{#1}}
\let\Oldsubsection\subsection
\renewcommand{\subsection}[1]{\FloatBarrier\Oldsubsection{#1}}
\let\Oldsubsubsection\subsubsection
\renewcommand{\subsubsection}[1]{\FloatBarrier\Oldsubsubsection{#1}}